\section{スキャンマッチング}

\subsection{概要}

スキャンマッチングとは,2つの点群を正しく照合できるような剛体変換$T \in \mathrm{SE}(3)$を求めることです.
ここで$T$は,回転行列$R \in \mathrm{SO}(3)$と並進ベクトル${\bf t} \in \mathbb{R}^{3}$を含みます.

今,点群$\mathcal{P} = ({\bf p}_{1}, ..., {\bf p}_{N})$と$\mathcal{Q} = ({\bf q}_{1}, ..., {\bf q}_{M})$を照合させることを考えます.
ただし${\bf p}, {\bf q} \in \mathbb{R}^{3}$です.
このとき,以下のコスト関数を考えます.
%
\begin{align}
  E = \sum_{i=1}^{N} \left\| {\bf q}_{i} - \left( R {\bf p}_{i} + {\bf t} \right) \right\|_{2}^{2}
  \label{eq:icp_cost_SO(3)}
\end{align}
%
ここで${\bf q}_{i}$は,${\bf p}_{i}$を剛体変換した点$R {\bf p}_{i} + {\bf t}$に最も近い$\mathcal{Q}$内の点です.
式(\ref{eq:icp_cost_SO(3)})は,以下のように書くことも可能です.
%
\begin{align}
  E = \sum_{i=1}^{N} \left\| {\bf q}_{i} - T {\bf p}_{i} \right\|_{2}^{2}
  \label{eq:icp_cost_SE(3)}
\end{align}
%
なおこの場合,${\bf p}, {\bf q} \in \mathbb{R}^{4}$となり,それぞれの4要素目には1が入ることになります.
そのため,$T {\bf p}$も4要素目が1の4次元ベクトルになりますが,${\bf q}$の4要素目も1となるため,${\bf q} - T {\bf p}$の4要素目は常に0になることととなり,結果としてコスト関数の値は式(\ref{eq:icp_cost_SO(3)})に示す値と同じになります.
なお以下では,${\bf q} - T {\bf p}$を誤差ベクトル${\bf e}$として定めます.

スキャンマッチングでは,以下に示す剛体変換を求めることを考えます.
%
\begin{align}
  T^{*} = \argmin_{T} E
  \label{eq:icp_scan_matching}
\end{align}
%
式(\ref{eq:icp_scan_matching})は,コスト関数$E$を最小化する姿勢$T^{*}$を求めるという意味になります.
この$T^{*}$は,一般的に反復処理を行うことで求めるられるため,Iterative Closest Points(ICP)スキャンマッチングとも呼ばれます.
なお式(\ref{eq:icp_cost_SE(3)})に示すコストを最小にするスキャンマッチングは,対応する点同士の距離を最小にするため,point-to-point ICPとも呼ばれます.

point-to-point ICPは一般的にノイズに脆弱であるといわれています.
そのため本書では,より頑健性の高い点と面の距離を最小化するpoint-to-plane ICPについて考えます.
point-to-plane ICPでは,以下のコスト関数を考えます.
%
\begin{align}
  E = \sum_{i=1}^{N} \left( {\bf n}_{i}^{\top} {\bf e}_{i} \right)^{2}
  \label{eq:point-to-plane_icp_cost_SE(3)}
\end{align}
%
ここで${\bf n}_{i}^{\top}$は,${\bf q}_{i}$の周辺の点を用いて計算した面の3次元空間内での法線ベクトルです.
なお,${\bf e}$が4次元ベクトルであるため${\bf n}$も4次元ベクトルとなりますが,${\bf e}$の4要素目は常に0であるため,${\bf n}$の4要素目がいくつであっても計算に違いは表れません.





\subsection{ヤコビアンの計算}

本書では,式(\ref{eq:point-to-plane_icp_cost_SE(3)})に示すコスト関数の最小化を行うために,ガウス・ニュートン法を用います.
そのために,残差$r = {\bf n}^{\top} {\bf e}$の姿勢$T$に関するヤコビアンを求めます.
このヤコビアンは,連鎖則を用いて以下のように計算できます.
%
\begin{align}
  \frac{ \partial r }{ \partial T } = \frac{ \partial r }{ \partial {\bf e} }
                                      \frac{ \partial {\bf e} }{ \partial T }
\end{align}
%
ここで,$\frac{ \partial r }{ \partial {\bf e} }$は明らかに${\bf n}^{\top}$です.
そのため,$\frac{ \partial {\bf e} }{ \partial T }$についてのみ詳細の計算方法を示します.

残差ベクトル${\bf e}$は,明らかに姿勢$T$に関する関数になっています.
そのため,次の微小変化を考え,ることでヤコビアン$J$を導出します.
%
\begin{align}
  {\bf e}(T \oplus \delta T) - {\bf e}(T) = 
  {\bf e}(\exp(\delta \boldsymbol \xi) T) - {\bf e}(T) \simeq
  J \delta \boldsymbol \xi
\end{align}
%
$\delta \boldsymbol \xi^{\top} = \left( \delta {\bf t}^{\top} ~ \delta \boldsymbol \theta^{\top} \right)^{\top} \in \mathfrak{se}(3)$
%
\begin{align}
  \begin{split}
    & {\bf q} - \exp(\delta \boldsymbol \xi) T {\bf p} - \left( {\bf q} - T {\bf p} \right) \\
    %
    = & - \left( \exp(\delta \boldsymbol \xi) - I_{4} \right) T {\bf p} \\
    %
    = & - \left( \left( \begin{matrix} I_{3} + [\delta \boldsymbol \theta]_{\times} & \delta {\bf t} \\ {\bf 0}^{\top} & 1 \end{matrix} \right) - I_{4} \right) \left( \begin{matrix} R {\bf p} + {\bf t} \\ 1 \end{matrix} \right) \\
    %
    = & - \left( \begin{matrix} [\delta \boldsymbol \theta]_{\times} & \delta {\bf t} \\ {\bf 0}^{\top} & 0 \end{matrix} \right) \left( \begin{matrix} {\bf p}' \\ 1 \end{matrix} \right) \\
    %
     = & - \left( \begin{matrix} [\delta \boldsymbol \theta]_{\times} {\bf p}' & \delta {\bf t} \\ {\bf 0}^{\top} & 0 \end{matrix} \right) \\
    %
     = & \left( \begin{matrix} [{\bf p}']_{\times} \delta \boldsymbol \theta & - \delta {\bf t} \\ {\bf 0}^{\top} & 0 \end{matrix} \right) \\
     %
     = & \left( \begin{matrix} -I_{3} & [{\bf p}']_{\times} \\ {\bf 0}^{\top} & {\bf 0}^{\top} \end{matrix} \right) \left( \begin{matrix} \delta {\bf t} \\ \delta \boldsymbol \theta \end{matrix} \right) =
     J \delta \boldsymbol \xi
  \end{split}
\end{align}
%
なお${\bf p}' = R {\bf p} + {\bf t}$と置き,$[\delta \boldsymbol \theta]_{\times} {\bf p}' = -[{\bf p}']_{\times} \delta \boldsymbol \theta$を用いました.
よって,残差に対するヤコビアンは以下となります.
%
\begin{align}
  \begin{split}
    \frac{ \partial r }{ \partial T } = & {\bf n}^{\top} \left( \begin{matrix} -I_{3} & [{\bf p}']_{\times} \\ {\bf 0}^{\top} & {\bf 0}^{\top} \end{matrix} \right) \\
    %
    = & \left( \begin{matrix} -{\bf n}^{\top} & {\bf n}^{\top} [{\bf p}']_{\times} \end{matrix} \right) \in \mathbb{R}^{1 \times 6}
  \end{split}
\end{align}
%
ただし最後の${\bf n}^{\top}$は3次元の法線ベクトルになります.

%
\begin{align}
  \begin{gathered}
    H = \sum_{i=1}^{N} J_{i}^{\top} J_{i} \\
    {\bf b} = \sum_{i=1}^{N} J_{i}^{\top} r_{i}
  \end{gathered}
\end{align}

